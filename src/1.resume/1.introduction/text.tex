\subsection{Introduction}



\subsubsection{Qu'est-ce qu'Internet ?}
Le réseau est composé de
\begin{itemize}
	\item \textbf{Système d'extrémité (end systems)} est un hôte, qui utilise des applications réseaux.
    \item \textbf{Liens de communication (communication links)} sont des point d'accès WiFi, câbles, ...
\end{itemize}
Pour communiquer entre eux, les systèmes périphériques utilisent différents \textit{protocoles}.
Un protocole est un algorithme décrivant la méthode utilisée pour l'envoi de données à
travers le réseau. Ils définissent plusieurs choses, comme le type de routage et de gestion
des congestions, càd qu'ils définissent par où acheminer les paquets pour limiter la charge,
éviter les obstacles, etc. Certains protocoles garantissent la fiabilité des transferts (tous
les paquets arrivent, et dans le bon ordre), d'autres font moins de vérifications mais sont
plus rapides, etc.



\subsubsection{Périphérie du réseau}
La périphérie du réseau (ou network edge) est l'ensemble des applications et systèmes
qui utilisent le réseau (par opposition aux liens et routeurs qui forment le coeur du réseau).
Elle est généralement organisée selon un de ces deux modèles :
Modèle client/serveur Une machine (le serveur) contient les informations et est toujours
présente sur le réseau. D'autres machines (les clients) se connectent à celle-ci
pour communiquer ; ils peuvent se connecter et se déconnecter à tout moment ; les
communications ne se font qu'entre clients et serveur.
Modèle peer-to-peer Chaque système est à la fois client et serveur. L'information est
décentralisée, et les différents systèmes communiquent entre eux directement.



\subsubsection{Aspects physiques}
Pour transmettre des données, on peut utiliser le réseau téléphonique classique. Pour
cela, on utilise un modem qui permet de (dé)coder l'information. Il existe plusieurs types de modulation :
Modulation d'amplitude (AM) Pour coder un signal binaire, on change l'amplitude
de l'onde selon que le bit soit à 1 ou à 0
Modulation de fréquence (FM) Pour coder un signal binaire, on change la fréquence
de l'onde selon que le bit soit à 1 ou à 0
De plus, pour augmenter le débit, on essaye de coder plusieurs bits d'un coup en combinant
ces techniques (on peut par exemple utiliser la phase et la fréquence pour coder
plusieurs bits d'un coup). On mesure alors le débit en baud, 1 baud étant 1 symbole/seconde
(càd 1 ensemble de bits par seconde)


\subsubsubsection{Limites de débit}
Avec le réseau téléphonique classique, le baud-rate (quantité de bauds par seconde)
est limitée à 4KHz, car c'est la fréquence prévue pour le téléphone (fréquence de la voix
humaine). On ne peut donc coder qu'un nombre limité de bits par bauds. De plus, il y a
une limite au maintien du signal : au minimum 1/2 période (sous ce seuil, on ne sait plus
interpréter le signal de manière univoque car on ne connaît p.e. pas l'amplitude). Enfin,
il y a une limitation quand à la quantité de bits codés par baud : on ne peut pas prendre
un amplitude trop grande pour coder les différents ensembles de bits, sans quoi on risque
de griller le fil de cuivre.
De plus, le data-rate (nombre de bits par seconde) est limité aussi, mais par la loi de
Shannon. En effet, moins le signal est maintenu longtemps, plus le bruit est important,
ce qui limite donc le débit.



\subsubsubsection{Réseau DSL}
Le réseau DSL utilise la même technique que la connexion dial-up, à la différence
qu'on utilise d'autres bandes de fréquences. Cela permet de pouvoir téléphoner et utiliser
le réseau simultanément, mais cela nécessite un équipement un peu plus important. On
sépare aussi le trafic "upstream" (vers le réseau) du trafic "downstream" (venant du
réseau), en donnant au premier une bande de fréquences moins importante, car on suppose
qu'on recevra plus de données qu'on en enverra la plupart du temps.



\subsubsubsection{Réseau cablé et fibre optique}
Cette fois, on n'utilise plus les fils de cuivre du réseau téléphonique comme support,
mais le cable TV et de la fibre optique (généralement un mélange des deux). A la différence
de l'ADSL, l'accès au réseau par câble est partagé, càd que les différents utilisateurs
partagent la même connexion, et donc les données et le débit. C'est fort pratique pour
la télévision (usage prévu de la fibre optique/coaxial), mais ça peut poser des problèmes
pour Internet (en termes de débit ou de protection des données).


\subsubsubsection{Types de câbles}
Historiquement, on utilise des câbles composés de deux fils de cuivre torsadés. Ils sont
torsadés pour limiter le courant induit (le passage de courant dans le fil crée un champ magnétique, et le passage de ce champ dans la boucle crée un autre courant, plus faible,
en sens inverse), et au plus ils sont torsadés, au plus cela a de l'incidence sur le débit
offert (on réduit le bruit, donc par la loi de Shannon on peut augmenter le signal).
Le câble coaxial fonctionne sur le même principe, à la différence que les conducteurs
sont concentriques (il y a un fil de cuivre à l'intérieur, puis une couche d'isolant, un autre fil de cuivre et le plastique qui entoure le câble).
Enfin, il y a la fibre optique, qui transporte l'information sous forme d'impulsions
lumineuses. Elles permettent le transfert de données à grande vitesse (vitesse de la lumière
dans le verre) et sont insensibles aux perturbations électromagnétiques (contrairement
aux fils de cuivre), mais il y a un problème lié à la réfraction de la lumière. En effet,
si on envoie une série de photons au même instant au départ, ils auront tous un angle
de départ différent, donc un angle d'incidence différent aux bords de la fibre. Du coup,
ils ne seront pas réfléchis avec le même angle, ce qui signifie que certains feront plus de
"rebonds" que d'autres, donc parcourront plus de distance, donc arriveront plus tard.
Au final, une impulsion lumineuse arrive dispersée, ce qui ralentit le débit puisqu'il faut
attendre l'arrivée de tous les photons avant d'envoyer l'impulsion suivante. Pour régler
ce problème, plusieurs types de fibre ont été inventés :
Fibre multimode Il s'agit du type de fibre qui ne résout pas le problème décrit : les
photons sont dispersés et arrivent avec un certain décalage ; tant pis, on fait avec
Fibre monomode Ce type de fibre résout le problème en laissant une zone de propagation
extrêmement étroite (2.4 $\mu$ m). De cette manière, les photons seront presque
parallèles à la fibre ; ils se réfléchis donc très peu sur les parois et seront peu
dispersés
Fibre multimode à coefficient variable Ce type de fibre ressemble à la fibre multimode,
à la différence que le coefficient de réfraction est différent en tout point de
la fibre. De cette manière, on s'arrange pour que les photons qui s'approchent du
bord de la fibre soient accélérés par rapport à ceux allant "tout droit". Au final,
certains photons auront donc fait plus de chemins que d'autres, mais ils arriveront
en même temps puisque leur vitesse varie.