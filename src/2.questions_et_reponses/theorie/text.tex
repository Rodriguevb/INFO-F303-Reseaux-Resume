\subsection{Théorie}


\newcommand{\questionA}{Expliquez la différence entre une paire de cuivre torsadée de catégorie 3 et une paire de catégorie 5. Laquelle permet un débit plus élevé et pourquoi ? } 
\newcommand{\questionB}{Expliquez la différence entre une fibre optique monomode et une fibre multimode. Laquelle permet un débit plus élevé et pourquoi ? }
\newcommand{\questionC}{Pourquoi utilise-t-on un modem pour transmettre de l'information numérique sur une ligne téléphonique ? Comment module-t-on le signal dans les modems « dial-up » les plus courants ? } 
\subsubsection[\questionA \questionB \questionC]{
\begin{enumerate}[(a)]
	\item \questionA
    \item \questionB
    \item \questionC
\end{enumerate}}
\color{reponse}
réponse réponse réponse réponse réponse réponse réponse réponse réponse réponse réponse réponse réponse réponse réponse réponse réponse réponse réponse réponse réponse réponse réponse réponse réponse réponse réponse réponse réponse réponse réponse réponse réponse réponse réponse réponse réponse réponse réponse réponse réponse réponse réponse réponse réponse réponse réponse réponse réponse réponse réponse réponse réponse 
\color{black}


\subsubsection{Définissez les différents types de « Resource Records » utilisés par le protocole DNS et expliquez leur rôle.}
\color{reponse}
réponse réponse réponse réponse réponse réponse réponse réponse réponse réponse réponse réponse réponse réponse réponse réponse réponse réponse réponse réponse réponse réponse réponse réponse réponse réponse réponse réponse réponse réponse réponse réponse réponse réponse réponse réponse réponse réponse réponse réponse réponse réponse réponse réponse réponse réponse réponse réponse réponse réponse réponse réponse réponse 
\color{black}


\newcommand{\question}{10 processus clients communiquent simultanément avec un processus serveur attaché au port 8000. } 
\renewcommand{\questionA}{Combien de sockets vont être ouverts par le serveur si les processus communiquent par UDP ? Pourquoi ? } 
\renewcommand{\questionB}{Même question s’ils communiquent par TCP. }

\subsubsection[\question \questionA \questionB]{\question
\begin{enumerate}[(a)]
	\item \questionA
    \item \questionB
\end{enumerate}}
\color{reponse}
réponse réponse réponse réponse réponse réponse réponse réponse réponse réponse réponse réponse réponse réponse réponse réponse réponse réponse réponse réponse réponse réponse réponse réponse réponse réponse réponse réponse réponse réponse réponse réponse réponse réponse réponse réponse réponse réponse réponse réponse réponse réponse réponse réponse réponse réponse réponse réponse réponse réponse réponse réponse réponse 
\color{black} 



\renewcommand{\questionA}{Expliquez le principe d’un protocole à fenêtre glissante SR (Selective Repeat). } 
\renewcommand{\questionB}{Quelle est la taille maximale de la fenêtre, si les trames sont numérotées modulo k ? Pourquoi ? }

\subsubsection[\questionA \questionB]{
\begin{enumerate}[(a)]
	\item \questionA
    \item \questionB
\end{enumerate}}
\color{reponse}
réponse réponse réponse réponse réponse réponse réponse réponse réponse réponse réponse réponse réponse réponse réponse réponse réponse réponse réponse réponse réponse réponse réponse réponse réponse réponse réponse réponse réponse réponse réponse réponse réponse réponse réponse réponse réponse réponse réponse réponse réponse réponse réponse réponse réponse réponse réponse réponse réponse réponse réponse réponse réponse 
\color{black}



\renewcommand{\questionA}{En première approximation, quels sont les 3 paramètres qui influencent le débit d’une connexion TCP ? Expliquez. }
\renewcommand{\questionB}{TCP garantit-il un partage équitable des ressources du réseau par les différentes connexions ? Pourquoi ? }

\subsubsection[\questionA \questionB]{
\begin{enumerate}[(a)]
	\item \questionA
    \item \questionB
\end{enumerate}}
\color{reponse}
réponse réponse réponse réponse réponse réponse réponse réponse réponse réponse réponse réponse réponse réponse réponse réponse réponse réponse réponse réponse réponse réponse réponse réponse réponse réponse réponse réponse réponse réponse réponse réponse réponse réponse réponse réponse réponse réponse réponse réponse réponse réponse réponse réponse réponse réponse réponse réponse réponse réponse réponse réponse réponse 
\color{black}



\renewcommand{\questionA}{Décrivez l’architecture générique d’un routeur et le rôle de chaque composant. } 
\renewcommand{\questionB}{Comment peut-on perdre des paquets dans les ports d’entrée ? }
\renewcommand{\questionC}{Comment peut-on perdre des paquets dans les ports de sortie ? }
\newcommand{\questionD}{Qu’est-ce que le blocage HOL ? }

\subsubsection[\questionA \questionB \questionC \questionD]{
\begin{enumerate}[(a)]
	\item \questionA
    \item \questionB
    \item \questionC
    \item \questionD
\end{enumerate}}
\color{reponse}
réponse réponse réponse réponse réponse réponse réponse réponse réponse réponse réponse réponse réponse réponse réponse réponse réponse réponse réponse réponse réponse réponse réponse réponse réponse réponse réponse réponse réponse réponse réponse réponse réponse réponse réponse réponse réponse réponse réponse réponse réponse réponse réponse réponse réponse réponse réponse réponse réponse réponse réponse réponse réponse 
\color{black}



\subsubsection{Décrivez le contenu des paquets de routage et leur méthode de diffusion dans le cas des protocoles à état de lien. En quelques mots, en quoi est-ce fondamentalement différent des protocoles à vecteur de distances ?}
\color{reponse}
réponse réponse réponse réponse réponse réponse réponse réponse réponse réponse réponse réponse réponse réponse réponse réponse réponse réponse réponse réponse réponse réponse réponse réponse réponse réponse réponse réponse réponse réponse réponse réponse réponse réponse réponse réponse réponse réponse réponse réponse réponse réponse réponse réponse réponse réponse réponse réponse réponse réponse réponse réponse réponse 
\color{black}


\renewcommand{\question}{Le protocole de routage interdomaine BGP est plus apparenté à la famille des protocoles de routage intradomaine à vecteur de distances (DV) qu’à celle des protocoles à état de lien (LS). } 
\renewcommand{\questionA}{Expliquez deux ressemblances importantes entre BGP et un protocole DV. } 
\renewcommand{\questionB}{Expliquez deux différences importantes entre BGP et un protocole DV, et leur raison d’être. }

\subsubsection[\question \questionA \questionB]{\question
\begin{enumerate}[(a)]
	\item \questionA
    \item \questionB
\end{enumerate}}
\color{reponse}
réponse réponse réponse réponse réponse réponse réponse réponse réponse réponse réponse réponse réponse réponse réponse réponse réponse réponse réponse réponse réponse réponse réponse réponse réponse réponse réponse réponse réponse réponse réponse réponse réponse réponse réponse réponse réponse réponse réponse réponse réponse réponse réponse réponse réponse réponse réponse réponse réponse réponse réponse réponse réponse 
\color{black}



\renewcommand{\questionA}{Qu'est-ce que le CSMA/CD ? En quoi améliore-t-il le CSMA ? } 
\renewcommand{\questionB}{Quelle contrainte le CSMA/CD introduit-il par rapport au CSMA ? Pourquoi ? }
\renewcommand{\questionC}{IEEE 802.3 (plus communément appelé Ethernet) est un protocole de type CSMA/CD dont la méthode d'accès a été améliorée. Quelle est cette amélioration ? } 
\renewcommand{\questionD}{Expliquez pourquoi, si l'on veut garder le même format de trame, la méthode CSMA/CD exige de raccourcir le réseau pour atteindre des débits plus élevés. Il est toutefois possible de ne pas respecter cette longueur maximale du réseau, qui devient très contraignante à haut débit. Dans quelles conditions ? } 
\subsubsection[\questionA \questionB \questionC \questionC]{
\begin{enumerate}[(a)]
	\item \questionA
    \item \questionB
    \item \questionC
    \item \questionD
\end{enumerate}}
\color{reponse}
réponse réponse réponse réponse réponse réponse réponse réponse réponse réponse réponse réponse réponse réponse réponse réponse réponse réponse réponse réponse réponse réponse réponse réponse réponse réponse réponse réponse réponse réponse réponse réponse réponse réponse réponse réponse réponse réponse réponse réponse réponse réponse réponse réponse réponse réponse réponse réponse réponse réponse réponse réponse réponse 
\color{black}



\subsubsection{On ne peut pas dire que les commutateurs Ethernet exécutent un protocole de routage (au sens de la couche 3), mais ils construisent toutefois des tables d’acheminement comme si un protocole de routage était à l’oeuvre. Expliquez comment ces tables sont construites, y compris quand plusieurs commutateurs sont interconnectés.}
\color{reponse}
réponse réponse réponse réponse réponse réponse réponse réponse réponse réponse réponse réponse réponse réponse réponse réponse réponse réponse réponse réponse réponse réponse réponse réponse réponse réponse réponse réponse réponse réponse réponse réponse réponse réponse réponse réponse réponse réponse réponse réponse réponse réponse réponse réponse réponse réponse réponse réponse réponse réponse réponse réponse réponse 
\color{black}



\renewcommand{\question}{Un chercheur connecte son ordinateur portable à un commutateur Ethernet de son département. Il démarre son browser pour afficher la page web de www.google.com. } 
\renewcommand{\questionA}{Identifiez les protocoles mis en oeuvre, et dans l’ordre chronologique, entre le moment où l’ordinateur se connecte et le moment où la page d’accueil de Google s’affiche. } 
\renewcommand{\questionB}{Précisez au passage le rôle de chaque protocole et décrivez-les succinctement. }
\subsubsection[\question \questionA \questionB]{\question
\begin{enumerate}[(a)]
	\item \questionA
    \item \questionB
\end{enumerate}}
\color{reponse}
réponse réponse réponse réponse réponse réponse réponse réponse réponse réponse réponse réponse réponse réponse réponse réponse réponse réponse réponse réponse réponse réponse réponse réponse réponse réponse réponse réponse réponse réponse réponse réponse réponse réponse réponse réponse réponse réponse réponse réponse réponse réponse réponse réponse réponse réponse réponse réponse réponse réponse réponse réponse réponse 
\color{black}



\renewcommand{\questionA}{Expliquez la dispersion de délai dans une fibre optique. } 
\renewcommand{\questionB}{Quelle en est la conséquence ? } 
\renewcommand{\questionC}{Dans quel type de fibre la rencontre-t-on ? }
\subsubsection[\questionA \questionB \questionC]{
\begin{enumerate}[(a)]
	\item \questionA
    \item \questionB
    \item \questionC
\end{enumerate}}
\color{reponse}
réponse réponse réponse réponse réponse réponse réponse réponse réponse réponse réponse réponse réponse réponse réponse réponse réponse réponse réponse réponse réponse réponse réponse réponse réponse réponse réponse réponse réponse réponse réponse réponse réponse réponse réponse réponse réponse réponse réponse réponse réponse réponse réponse réponse réponse réponse réponse réponse réponse réponse réponse réponse réponse 
\color{black}



\renewcommand{\questionA}{Définissez les différents types de « Resource Records (RR)» utilisés par le protocole DNS et expliquez leur rôle. } 
\renewcommand{\questionB}{Donnez le scénario d’échange de messages DNS, par la méthode itérative, permettant à un client de trouver l’adresse IP d’un serveur web dont l’URL est www.company.com, \textit{en indiquant les RR présents dans ces messages}. On supposera que les caches DNS sont vides. }
\subsubsection[\questionA \questionB]{
\begin{enumerate}[(a)]
	\item \questionA
    \item \questionB
\end{enumerate}}
\color{reponse}
réponse réponse réponse réponse réponse réponse réponse réponse réponse réponse réponse réponse réponse réponse réponse réponse réponse réponse réponse réponse réponse réponse réponse réponse réponse réponse réponse réponse réponse réponse réponse réponse réponse réponse réponse réponse réponse réponse réponse réponse réponse réponse réponse réponse réponse réponse réponse réponse réponse réponse réponse réponse réponse 
\color{black}



\renewcommand{\questionA}{Expliquez le principe d’un protocole à fenêtre glissante GBN (Go-back N). } 
\renewcommand{\questionB}{Quelle est la taille maximale de la fenêtre de l’émetteur, si les trames sont numérotées modulo k ? Pourquoi ? } 
\renewcommand{\questionC}{Citez et expliquez 4 différences apportées par le protocole SR (Selective Repeat). }
\subsubsection[\questionA \questionB \questionC]{
\begin{enumerate}[(a)]
	\item \questionA
    \item \questionB
    \item \questionC
\end{enumerate}}
\color{reponse}
réponse réponse réponse réponse réponse réponse réponse réponse réponse réponse réponse réponse réponse réponse réponse réponse réponse réponse réponse réponse réponse réponse réponse réponse réponse réponse réponse réponse réponse réponse réponse réponse réponse réponse réponse réponse réponse réponse réponse réponse réponse réponse réponse réponse réponse réponse réponse réponse réponse réponse réponse réponse réponse 
\color{black}



\renewcommand{\questionA}{Expliquez l’établissement de connexion « 3-way handshake » utilisé dans le protocole de transport TCP, en indiquant les paramètres importants présents dans les échanges et leurs rôles. }
\renewcommand{\questionB}{Expliquez avec l’aide d’un exemple pourquoi un « 2-way handshake » ne serait pas suffisant. }
\subsubsection[\questionA \questionB]{
\begin{enumerate}[(a)]
	\item \questionA
    \item \questionB
\end{enumerate}}
\color{reponse}
réponse réponse réponse réponse réponse réponse réponse réponse réponse réponse réponse réponse réponse réponse réponse réponse réponse réponse réponse réponse réponse réponse réponse réponse réponse réponse réponse réponse réponse réponse réponse réponse réponse réponse réponse réponse réponse réponse réponse réponse réponse réponse réponse réponse réponse réponse réponse réponse réponse réponse réponse réponse réponse 
\color{black}



\renewcommand{\questionA}{Comment l’émetteur TCP détecte-t-il une congestion ? } 
\renewcommand{\questionB}{Décrivez le mécanisme de contrôle de congestion de TCP. }
\renewcommand{\questionC}{Quelle distinction TCP fait-il entre congestion légère et congestion sévère ? Comment réagit-il dans chaque cas ? } 
\renewcommand{\questionD}{Si on néglige les effets du contrôle de flux, ce contrôle de congestion détermine largement le débit moyen d’une connexion TCP. Quand plusieurs connexions TCP sont en compétition, se partagentelles la bande passante disponible de façon équitable. Expliquez. } 
\subsubsection[\questionA \questionB \questionC \questionC]{
\begin{enumerate}[(a)]
	\item \questionA
    \item \questionB
    \item \questionC
    \item \questionD
\end{enumerate}}
\color{reponse}
réponse réponse réponse réponse réponse réponse réponse réponse réponse réponse réponse réponse réponse réponse réponse réponse réponse réponse réponse réponse réponse réponse réponse réponse réponse réponse réponse réponse réponse réponse réponse réponse réponse réponse réponse réponse réponse réponse réponse réponse réponse réponse réponse réponse réponse réponse réponse réponse réponse réponse réponse réponse réponse 
\color{black}



\renewcommand{\questionA}{Enoncez les différents types de matrice de commutation (« switch fabric ») rencontrées dans les routeurs, ainsi que leurs avantages/inconvénients respectifs. } 
\renewcommand{\questionB}{Expliquez la raison d’être et l’inconvénient potentiel d’une bufferisation au niveau des ports d’entrées. } 
\renewcommand{\questionC}{Expliquez la raison d’être d’une bufferisation au niveau des ports de sortie. }
\subsubsection[\questionA \questionB \questionC]{
\begin{enumerate}[(a)]
	\item \questionA
    \item \questionB
    \item \questionC
\end{enumerate}}
\color{reponse}
réponse réponse réponse réponse réponse réponse réponse réponse réponse réponse réponse réponse réponse réponse réponse réponse réponse réponse réponse réponse réponse réponse réponse réponse réponse réponse réponse réponse réponse réponse réponse réponse réponse réponse réponse réponse réponse réponse réponse réponse réponse réponse réponse réponse réponse réponse réponse réponse réponse réponse réponse réponse réponse 
\color{black}



\renewcommand{\questionA}{Expliquez le principe du « Longest Prefix Match » lors de l’acheminement de paquets IP. }
\renewcommand{\questionB}{Quel est son intérêt ? }
\subsubsection[\questionA \questionB]{
\begin{enumerate}[(a)]
	\item \questionA
    \item \questionB
\end{enumerate}}
\color{reponse}
réponse réponse réponse réponse réponse réponse réponse réponse réponse réponse réponse réponse réponse réponse réponse réponse réponse réponse réponse réponse réponse réponse réponse réponse réponse réponse réponse réponse réponse réponse réponse réponse réponse réponse réponse réponse réponse réponse réponse réponse réponse réponse réponse réponse réponse réponse réponse réponse réponse réponse réponse réponse réponse 
\color{black}



\renewcommand{\question}{Le protocole de routage interdomaine BGP est plus apparenté à la famille des protocoles de routage intradomaine à vecteur de distances (DV) qu’à celle des protocoles à état de lien (LS). }
\renewcommand{\questionA}{Expliquez deux ressemblances importantes entre BGP et un protocole DV. }
\renewcommand{\questionB}{Expliquez deux différences importantes entre BGP et un protocole DV, et leur raison d’être. }
\subsubsection[\question \questionA \questionB]{\question
\begin{enumerate}[(a)]
	\item \questionA
    \item \questionB
\end{enumerate}}
\color{reponse}
réponse réponse réponse réponse réponse réponse réponse réponse réponse réponse réponse réponse réponse réponse réponse réponse réponse réponse réponse réponse réponse réponse réponse réponse réponse réponse réponse réponse réponse réponse réponse réponse réponse réponse réponse réponse réponse réponse réponse réponse réponse réponse réponse réponse réponse réponse réponse réponse réponse réponse réponse réponse réponse 
\color{black}



\renewcommand{\questionA}{Expliquez le rôle et le principe général des codes détecteurs d’erreur. } 
\renewcommand{\questionB}{Pourquoi ne peuvent-ils être efficaces à 100\% ? } 
\renewcommand{\questionC}{Donnez un exemple de code détecteur d’erreur plus élaboré que le bit de parité, et expliquez son principe. }
\subsubsection[\questionA \questionB \questionC]{
\begin{enumerate}[(a)]
	\item \questionA
    \item \questionB
    \item \questionC
\end{enumerate}}
\color{reponse}
réponse réponse réponse réponse réponse réponse réponse réponse réponse réponse réponse réponse réponse réponse réponse réponse réponse réponse réponse réponse réponse réponse réponse réponse réponse réponse réponse réponse réponse réponse réponse réponse réponse réponse réponse réponse réponse réponse réponse réponse réponse réponse réponse réponse réponse réponse réponse réponse réponse réponse réponse réponse réponse 
\color{black}



\renewcommand{\questionA}{Expliquez le principe du multiplexage en longueur d’onde (WDM). Quel est son intérêt ? }
\renewcommand{\questionB}{Comparez WDM aux techniques classiques de multiplexage TDM et FDM.}
\subsubsection[\questionA \questionB]{
\begin{enumerate}[(a)]
	\item \questionA
    \item \questionB
\end{enumerate}}
\color{reponse}
réponse réponse réponse réponse réponse réponse réponse réponse réponse réponse réponse réponse réponse réponse réponse réponse réponse réponse réponse réponse réponse réponse réponse réponse réponse réponse réponse réponse réponse réponse réponse réponse réponse réponse réponse réponse réponse réponse réponse réponse réponse réponse réponse réponse réponse réponse réponse réponse réponse réponse réponse réponse réponse 
\color{black}



\renewcommand{\question}{Vous créez votre entreprise « MeMyself\&I » et vous obtenez le nom de domaine « memyselfandi.com ». Vous souhaitez déployer votre propre serveur DNS pour ce domaine (dns.memyselfandi.com, 111.111.111.111), ainsi qu’un serveur Web  www.memyselfandi.com, 111.111.111.112). }
\renewcommand{\questionA}{Quelles informations doivent être ajoutées dans la hiérarchie DNS et à quel niveau ? Soyez précis. }
\renewcommand{\questionB}{Donnez un scénario typique d’échange de messages DNS permettant à un client de trouver l’adresse IP de votre serveur web, en précisant bien les éléments importants des messages DNS. On supposera que les caches DNS sont vides.}
\subsubsection[\question \questionA \questionB]{\question
\begin{enumerate}[(a)]
	\item \questionA
    \item \questionB
\end{enumerate}}
\color{reponse}
réponse réponse réponse réponse réponse réponse réponse réponse réponse réponse réponse réponse réponse réponse réponse réponse réponse réponse réponse réponse réponse réponse réponse réponse réponse réponse réponse réponse réponse réponse réponse réponse réponse réponse réponse réponse réponse réponse réponse réponse réponse réponse réponse réponse réponse réponse réponse réponse réponse réponse réponse réponse réponse 
\color{black}



\renewcommand{\questionA}{Pourquoi la couche de transport (UDP et TCP) comporte-t-elle une fonction de démultiplexage ? }
\renewcommand{\questionB}{Décrivez les techniques de démultiplexage effectuées par UDP et TCP en mettant bien en évidence leurs différences ?}
\subsubsection[\questionA \questionB]{
\begin{enumerate}[(a)]
	\item \questionA
    \item \questionB
\end{enumerate}}
\color{reponse}
réponse réponse réponse réponse réponse réponse réponse réponse réponse réponse réponse réponse réponse réponse réponse réponse réponse réponse réponse réponse réponse réponse réponse réponse réponse réponse réponse réponse réponse réponse réponse réponse réponse réponse réponse réponse réponse réponse réponse réponse réponse réponse réponse réponse réponse réponse réponse réponse réponse réponse réponse réponse réponse 
\color{black}



\renewcommand{\questionA}{Donnez 4 éléments majeurs des protocoles « Go-Back-N » et « Selective Repeat » qui permettent
de les différencier. } 
\renewcommand{\questionB}{Pour chacun de ces éléments pris indépendamment, indiquez si TCP s’apparente davantage à l’un d’eux. Expliquez. } 
\renewcommand{\questionC}{Quelle optimisation supplémentaire, liée au contrôle d’erreur, TCP y apporte-t-il ?}
\subsubsection[\questionA \questionB \questionC]{
\begin{enumerate}[(a)]
	\item \questionA
    \item \questionB
    \item \questionC
\end{enumerate}}
\color{reponse}
réponse réponse réponse réponse réponse réponse réponse réponse réponse réponse réponse réponse réponse réponse réponse réponse réponse réponse réponse réponse réponse réponse réponse réponse réponse réponse réponse réponse réponse réponse réponse réponse réponse réponse réponse réponse réponse réponse réponse réponse réponse réponse réponse réponse réponse réponse réponse réponse réponse réponse réponse réponse réponse 
\color{black}



\subsubsection{Expliquez le principe de NAT et la structure d’une table NAT.}
\color{reponse}
réponse réponse réponse réponse réponse réponse réponse réponse réponse réponse réponse réponse réponse réponse réponse réponse réponse réponse réponse réponse réponse réponse réponse réponse réponse réponse réponse réponse réponse réponse réponse réponse réponse réponse réponse réponse réponse réponse réponse réponse réponse réponse réponse réponse réponse réponse réponse réponse réponse réponse réponse réponse réponse 
\color{black}



\subsubsection{Quand des flux TCP et UDP partagent un même lien congestionné, comment réagissent ces deux types de flux et quelles en sont les conséquences ?}
\color{reponse}
réponse réponse réponse réponse réponse réponse réponse réponse réponse réponse réponse réponse réponse réponse réponse réponse réponse réponse réponse réponse réponse réponse réponse réponse réponse réponse réponse réponse réponse réponse réponse réponse réponse réponse réponse réponse réponse réponse réponse réponse réponse réponse réponse réponse réponse réponse réponse réponse réponse réponse réponse réponse réponse 
\color{black}



\renewcommand{\questionA}{Nommez et expliquez succinctement les 2 grandes familles de protocoles de routage intradomaine (IGP) en insistant sur leurs différences. }
\renewcommand{\questionB}{Expliquez en quoi et pourquoi le protocole de routage interdomaine de l’Internet (BGP) est différent des protocoles de routage intradomaine (IGP) déployés dans les divers systèmes autonomes (AS) qui composent l’Internet.}
\subsubsection[\questionA \questionB]{
\begin{enumerate}[(a)]
	\item \questionA
    \item \questionB
\end{enumerate}}
\color{reponse}
réponse réponse réponse réponse réponse réponse réponse réponse réponse réponse réponse réponse réponse réponse réponse réponse réponse réponse réponse réponse réponse réponse réponse réponse réponse réponse réponse réponse réponse réponse réponse réponse réponse réponse réponse réponse réponse réponse réponse réponse réponse réponse réponse réponse réponse réponse réponse réponse réponse réponse réponse réponse réponse 
\color{black}



\subsubsection{Expliquez comment un routeur construit les entrées de sa table d’acheminement pour les préfixes IP extérieurs à son domaine.}
\color{reponse}
réponse réponse réponse réponse réponse réponse réponse réponse réponse réponse réponse réponse réponse réponse réponse réponse réponse réponse réponse réponse réponse réponse réponse réponse réponse réponse réponse réponse réponse réponse réponse réponse réponse réponse réponse réponse réponse réponse réponse réponse réponse réponse réponse réponse réponse réponse réponse réponse réponse réponse réponse réponse réponse 
\color{black}



\renewcommand{\questionA}{Décrivez le protocole CSMA. } 
\renewcommand{\questionB}{Pourquoi et comment a-t-il été amélioré ? } 
\renewcommand{\questionC}{Citez les paramètres qui caractérisent un réseau CSMA. Quelle relation entre ces paramètres faut-il viser pour que le réseau CSMA ait des performances acceptables ? Expliquez.}
\subsubsection[\questionA \questionB \questionC]{
\begin{enumerate}[(a)]
	\item \questionA
    \item \questionB
    \item \questionC
\end{enumerate}}
\color{reponse}
réponse réponse réponse réponse réponse réponse réponse réponse réponse réponse réponse réponse réponse réponse réponse réponse réponse réponse réponse réponse réponse réponse réponse réponse réponse réponse réponse réponse réponse réponse réponse réponse réponse réponse réponse réponse réponse réponse réponse réponse réponse réponse réponse réponse réponse réponse réponse réponse réponse réponse réponse réponse réponse 
\color{black}



\renewcommand{\question}{Considérez 3 réseaux Ethernet ($N_1$, $N_2$ et $N_3$), un commutateur Ethernet ($C$) et un routeur ($R$) interconnectés selon une topologie en ligne $N_1$-$C$-$N_2$-$R$-$N_3$. Une station $H_A$ (d’adresse $IP_A$) est attachée au réseau $N_1$ (par l’adresse $MAC_A$) et une station $H_B$ (d’adresse $IP_B$) est attachée au réseau $N_3$ (par l’adresse $MAC_B$). $C$ a deux adresses $MAC$ : $MAC_{11}$ sur $N_1$ et $MAC_{12}$ sur $N_2$. $R$ a deux adresses $MAC$ et deux adresses $IP$ : $MAC_{22}$ et $IP_2$ sur $N_2$ et $MAC_{23}$ et $IP_3$ sur $N_3$. }
\renewcommand{\questionA}{Dessinez la configuration. $H_A$ envoie un paquet $IP$ à $H_B$. Si l’on suppose que les correspondances entre adresses $IP$ et $MAC$ sont connues de tous, décrivez les trois trames qui circulent respectivement sur les réseaux $N_1$, $N_2$ et $N_3$ en vous limitant aux champs d’adresses des trames et aux champs d’adresses et de $TTL$ ($T$ime $T$o $L$ive) du paquet $IP$ contenu dans la trame. Justifiez. }
\renewcommand{\questionB}{Par quel protocole les correspondances entre adresses $IP$ et $MAC$ ont-elles été découvertes ? Décrivez les échanges de ce protocole qui réalisent les mises en correspondance nécessaires lorsque $H_A$ envoie son paquet $IP$ à $H_B$. Mentionnez toutes les adresses présentes dans les messages échangés.}
\subsubsection[\texorpdfstring{\question \questionA \questionB}.]{\question
\begin{enumerate}[(a)]
	\item \questionA
    \item \questionB
\end{enumerate}}
\color{reponse}
réponse réponse réponse réponse réponse réponse réponse réponse réponse réponse réponse réponse réponse réponse réponse réponse réponse réponse réponse réponse réponse réponse réponse réponse réponse réponse réponse réponse réponse réponse réponse réponse réponse réponse réponse réponse réponse réponse réponse réponse réponse réponse réponse réponse réponse réponse réponse réponse réponse réponse réponse réponse réponse 
\color{black}



\subsubsection{Citez une fonction majeure de chacune des 5 couches de la pile de protocoles Internet.}
\color{reponse}
réponse réponse réponse réponse réponse réponse réponse réponse réponse réponse réponse réponse réponse réponse réponse réponse réponse réponse réponse réponse réponse réponse réponse réponse réponse réponse réponse réponse réponse réponse réponse réponse réponse réponse réponse réponse réponse réponse réponse réponse réponse réponse réponse réponse réponse réponse réponse réponse réponse réponse réponse réponse réponse 
\color{black}



\renewcommand{\questionA}{Pourquoi est-il plus difficile de fixer la durée du timer de retransmission de TCP que celle du timer
de retransmission d’un protocole de liaison de donnée ? } 
\renewcommand{\questionB}{Comment fixe-t-on la durée du timer de retransmission de TCP ?}
\subsubsection[\questionA \questionB]{
\begin{enumerate}[(a)]
	\item \questionA
    \item \questionB
\end{enumerate}}
\color{reponse}
réponse réponse réponse réponse réponse réponse réponse réponse réponse réponse réponse réponse réponse réponse réponse réponse réponse réponse réponse réponse réponse réponse réponse réponse réponse réponse réponse réponse réponse réponse réponse réponse réponse réponse réponse réponse réponse réponse réponse réponse réponse réponse réponse réponse réponse réponse réponse réponse réponse réponse réponse réponse réponse 
\color{black}



\subsubsection{Expliquez la raison d’être des protocoles DHCP et NAT, et expliquez leur fonctionnement à l’aide de scénarios typiques.}
\color{reponse}
réponse réponse réponse réponse réponse réponse réponse réponse réponse réponse réponse réponse réponse réponse réponse réponse réponse réponse réponse réponse réponse réponse réponse réponse réponse réponse réponse réponse réponse réponse réponse réponse réponse réponse réponse réponse réponse réponse réponse réponse réponse réponse réponse réponse réponse réponse réponse réponse réponse réponse réponse réponse réponse 
\color{black}



\renewcommand{\questionA}{Expliquez comment les commutateurs Ethernet apprennent où se trouvent les stations et par quel
type d’adresse ils les identifient. } 
\renewcommand{\questionB}{Comment les pannes de stations ou leur mobilité sont-elles prises en compte ? } 
\renewcommand{\questionC}{En quelques mots, quelle contrainte topologique doit être respectée pour que cet apprentissage
fonctionne, et comment la réalise-t-on ?}
\subsubsection[\questionA \questionB \questionC]{
\begin{enumerate}[(a)]
	\item \questionA
    \item \questionB
    \item \questionC
\end{enumerate}}
\color{reponse}
réponse réponse réponse réponse réponse réponse réponse réponse réponse réponse réponse réponse réponse réponse réponse réponse réponse réponse réponse réponse réponse réponse réponse réponse réponse réponse réponse réponse réponse réponse réponse réponse réponse réponse réponse réponse réponse réponse réponse réponse réponse réponse réponse réponse réponse réponse réponse réponse réponse réponse réponse réponse réponse 
\color{black}



\subsubsection{Citez et définissez les différentes sources de délai que subit un paquet dans un réseau datagramme.}
\color{reponse}
réponse réponse réponse réponse réponse réponse réponse réponse réponse réponse réponse réponse réponse réponse réponse réponse réponse réponse réponse réponse réponse réponse réponse réponse réponse réponse réponse réponse réponse réponse réponse réponse réponse réponse réponse réponse réponse réponse réponse réponse réponse réponse réponse réponse réponse réponse réponse réponse réponse réponse réponse réponse réponse 
\color{black}



\renewcommand{\questionA}{Décrivez sommairement le fonctionnement du système DNS. } 
\renewcommand{\questionB}{Comparez les deux modes de fonctionnement du protocole (avantages et inconvénients).}
\subsubsection[\questionA \questionB]{
\begin{enumerate}[(a)]
	\item \questionA
    \item \questionB
\end{enumerate}}
\color{reponse}
réponse réponse réponse réponse réponse réponse réponse réponse réponse réponse réponse réponse réponse réponse réponse réponse réponse réponse réponse réponse réponse réponse réponse réponse réponse réponse réponse réponse réponse réponse réponse réponse réponse réponse réponse réponse réponse réponse réponse réponse réponse réponse réponse réponse réponse réponse réponse réponse réponse réponse réponse réponse réponse 
\color{black}



\renewcommand{\questionA}{Expliquer les principes de la programmation socket donnant accès aux services TCP et UDP. } 
\renewcommand{\questionB}{Quelles sont les différences importantes entre ces deux API ? } 
\renewcommand{\questionC}{Dans une entité de transport, comment les sockets TCP et UDP sont-ils identifiés ? Pourquoi ?}
\subsubsection[\questionA \questionB \questionC]{
\begin{enumerate}[(a)]
	\item \questionA
    \item \questionB
    \item \questionC
\end{enumerate}}
\color{reponse}
réponse réponse réponse réponse réponse réponse réponse réponse réponse réponse réponse réponse réponse réponse réponse réponse réponse réponse réponse réponse réponse réponse réponse réponse réponse réponse réponse réponse réponse réponse réponse réponse réponse réponse réponse réponse réponse réponse réponse réponse réponse réponse réponse réponse réponse réponse réponse réponse réponse réponse réponse réponse réponse 
\color{black}



\renewcommand{\questionA}{Dans un protocole de transport, si l’on numérote les segments modulo 2, montrez par un contreexemple
qu’il est également nécessaire de numéroter les acquits pour assurer la fiabilité du transfert. } 
\renewcommand{\questionB}{Dans quelle(s) situation(s) le protocole à bit alterné est-il quasiment aussi efficace qu’un protocole
à grande fenêtre glissante? Expliquez. }
\subsubsection[\questionA \questionB]{
\begin{enumerate}[(a)]
	\item \questionA
    \item \questionB
\end{enumerate}}
\color{reponse}
réponse réponse réponse réponse réponse réponse réponse réponse réponse réponse réponse réponse réponse réponse réponse réponse réponse réponse réponse réponse réponse réponse réponse réponse réponse réponse réponse réponse réponse réponse réponse réponse réponse réponse réponse réponse réponse réponse réponse réponse réponse réponse réponse réponse réponse réponse réponse réponse réponse réponse réponse réponse réponse 
\color{black}



\renewcommand{\questionA}{Expliquez les circonstances dans lesquelles l’émetteur TCP peut recevoir trois doublons d’acquits
venant du récepteur TCP. } 
\renewcommand{\questionB}{Décrivez deux actions importantes de l’émetteur TCP lorsque cela se produit et expliquez-en les
raisons. }
\subsubsection[\questionA \questionB]{
\begin{enumerate}[(a)]
	\item \questionA
    \item \questionB
\end{enumerate}}
\color{reponse}
réponse réponse réponse réponse réponse réponse réponse réponse réponse réponse réponse réponse réponse réponse réponse réponse réponse réponse réponse réponse réponse réponse réponse réponse réponse réponse réponse réponse réponse réponse réponse réponse réponse réponse réponse réponse réponse réponse réponse réponse réponse réponse réponse réponse réponse réponse réponse réponse réponse réponse réponse réponse réponse 
\color{black}



\renewcommand{\questionA}{Expliquez le principe général du contrôle de \textit{flux} de TCP. } 
\renewcommand{\questionB}{Expliquez deux mécanismes associés ayant pour but de permettre à TCP de s’adapter aux spécificités des applications ou de se protéger vis-à-vis de celles-ci. }
\subsubsection[\questionA \questionB]{
\begin{enumerate}[(a)]
	\item \questionA
    \item \questionB
\end{enumerate}}
\color{reponse}
réponse réponse réponse réponse réponse réponse réponse réponse réponse réponse réponse réponse réponse réponse réponse réponse réponse réponse réponse réponse réponse réponse réponse réponse réponse réponse réponse réponse réponse réponse réponse réponse réponse réponse réponse réponse réponse réponse réponse réponse réponse réponse réponse réponse réponse réponse réponse réponse réponse réponse réponse réponse réponse 
\color{black}



\subsubsection{Combien d’adresses IP doit-on attribuer à un routeur ? Pourquoi ?}
\color{reponse}
réponse réponse réponse réponse réponse réponse réponse réponse réponse réponse réponse réponse réponse réponse réponse réponse réponse réponse réponse réponse réponse réponse réponse réponse réponse réponse réponse réponse réponse réponse réponse réponse réponse réponse réponse réponse réponse réponse réponse réponse réponse réponse réponse réponse réponse réponse réponse réponse réponse réponse réponse réponse réponse 
\color{black}



\renewcommand{\questionA}{Considérez un protocole de routage à états de liens (link state). Décrivez le contenu des paquets de
routage, expliquez le rôle de chaque champ, et décrivez la méthode de diffusion des paquets. } 
\renewcommand{\questionB}{En quelques mots, en quoi est-ce fondamentalement différent des protocoles à vecteur de
distances ? }
\subsubsection[\questionA \questionB]{
\begin{enumerate}[(a)]
	\item \questionA
    \item \questionB
\end{enumerate}}
\color{reponse}
réponse réponse réponse réponse réponse réponse réponse réponse réponse réponse réponse réponse réponse réponse réponse réponse réponse réponse réponse réponse réponse réponse réponse réponse réponse réponse réponse réponse réponse réponse réponse réponse réponse réponse réponse réponse réponse réponse réponse réponse réponse réponse réponse réponse réponse réponse réponse réponse réponse réponse réponse réponse réponse 
\color{black}



\renewcommand{\questionA}{Décrivez les principes du protocole de routage inter-domaine BGP. } 
\renewcommand{\questionB}{Expliquez comment BGP permet à un réseau périphérique (« stub ») multi-connecté (« multihomed
») de ne pas accepter du trafic de transit. }
\subsubsection[\questionA \questionB]{
\begin{enumerate}[(a)]
	\item \questionA
    \item \questionB
\end{enumerate}}
\color{reponse}
réponse réponse réponse réponse réponse réponse réponse réponse réponse réponse réponse réponse réponse réponse réponse réponse réponse réponse réponse réponse réponse réponse réponse réponse réponse réponse réponse réponse réponse réponse réponse réponse réponse réponse réponse réponse réponse réponse réponse réponse réponse réponse réponse réponse réponse réponse réponse réponse réponse réponse réponse réponse réponse 
\color{black}



\subsubsection{Sachant que la couche de transport est équipée de mécanismes (Cf. TCP) pour récupérer les erreurs de
bout-en-bout, pourquoi la couche de liaison de données implémente-t-elle aussi toute une série de
fonctions de ce type, comme la détection d’erreurs, voire même la retransmission de trames erronées
dans certains cas.}
\color{reponse}
réponse réponse réponse réponse réponse réponse réponse réponse réponse réponse réponse réponse réponse réponse réponse réponse réponse réponse réponse réponse réponse réponse réponse réponse réponse réponse réponse réponse réponse réponse réponse réponse réponse réponse réponse réponse réponse réponse réponse réponse réponse réponse réponse réponse réponse réponse réponse réponse réponse réponse réponse réponse réponse 
\color{black}



\renewcommand{\questionA}{Dans un réseau local composé de plusieurs segments Ethernet interconnectés par des commutateurs Ethernet, un ordinateur peut-il conserver son adresse IP si on le change de segment ? Pourquoi ? } 
\renewcommand{\questionB}{En est-il de même si les segments sont interconnectés par des routeurs ? Pourquoi ? } 
\renewcommand{\questionC}{Pourquoi est-il plus intéressant d’interconnecter des segments Ethernet par des commutateurs Ethernet plutôt que par des hubs ?}
\subsubsection[\questionA \questionB \questionC]{
\begin{enumerate}[(a)]
	\item \questionA
    \item \questionB
    \item \questionC
\end{enumerate}}
\color{reponse}
réponse réponse réponse réponse réponse réponse réponse réponse réponse réponse réponse réponse réponse réponse réponse réponse réponse réponse réponse réponse réponse réponse réponse réponse réponse réponse réponse réponse réponse réponse réponse réponse réponse réponse réponse réponse réponse réponse réponse réponse réponse réponse réponse réponse réponse réponse réponse réponse réponse réponse réponse réponse réponse 
\color{black}



\renewcommand{\questionA}{Expliquez la différence entre une fibre optique multimode et une fibre monomode. } 
\renewcommand{\questionB}{Laquelle permet un débit plus élevé ? Pourquoi ? }
\renewcommand{\questionC}{Expliquez le multiplexage en longueur d’onde (WDM). Quel est son intérêt ? } 
\renewcommand{\questionD}{Comparez TDM, FDM et WDM. } 
\subsubsection[\questionA \questionB \questionC \questionC]{
\begin{enumerate}[(a)]
	\item \questionA
    \item \questionB
    \item \questionC
    \item \questionD
\end{enumerate}}
\color{reponse}
réponse réponse réponse réponse réponse réponse réponse réponse réponse réponse réponse réponse réponse réponse réponse réponse réponse réponse réponse réponse réponse réponse réponse réponse réponse réponse réponse réponse réponse réponse réponse réponse réponse réponse réponse réponse réponse réponse réponse réponse réponse réponse réponse réponse réponse réponse réponse réponse réponse réponse réponse réponse réponse 
\color{black}



\renewcommand{\questionA}{Quel mécanisme est utilisé par un serveur Web pour conserver de l’état relatif aux usagers ?
Expliquez le principe en l’illustrant sur un scénario. } 
\renewcommand{\questionB}{Expliquer le fonctionnement de HTTP avec proxy-cache à partir d’un scénario impliquant le
client, le serveur et le proxy. Expliquez le gain d’efficacité lorsque l’objet est en cache. }
\subsubsection[\questionA \questionB]{
\begin{enumerate}[(a)]
	\item \questionA
    \item \questionB
\end{enumerate}}
\color{reponse}
réponse réponse réponse réponse réponse réponse réponse réponse réponse réponse réponse réponse réponse réponse réponse réponse réponse réponse réponse réponse réponse réponse réponse réponse réponse réponse réponse réponse réponse réponse réponse réponse réponse réponse réponse réponse réponse réponse réponse réponse réponse réponse réponse réponse réponse réponse réponse réponse réponse réponse réponse réponse réponse 
\color{black}



\renewcommand{\questionA}{Dans un protocole de transport, si l’on numérote les segments modulo 2, montrez par un contreexemple
qu’il est également nécessaire de numéroter les acquits pour assurer la fiabilité du transfert. } 
\renewcommand{\questionB}{Dans quelle(s) situation(s) le protocole à bit alterné est-il quasiment aussi efficace qu’un protocole
à grande fenêtre glissante ? Expliquez. }
\subsubsection[\questionA \questionB]{
\begin{enumerate}[(a)]
	\item \questionA
    \item \questionB
\end{enumerate}}
\color{reponse}
réponse réponse réponse réponse réponse réponse réponse réponse réponse réponse réponse réponse réponse réponse réponse réponse réponse réponse réponse réponse réponse réponse réponse réponse réponse réponse réponse réponse réponse réponse réponse réponse réponse réponse réponse réponse réponse réponse réponse réponse réponse réponse réponse réponse réponse réponse réponse réponse réponse réponse réponse réponse réponse 
\color{black}



\subsubsection{Dans les protocoles à fenêtre glissante de type « selective repeat », quelles sont les relations qui sont
satisfaites à tout instant entre les quatre valeurs suivantes : les bords inférieurs et supérieurs des
fenêtres de l’émetteur et du récepteur ? Justifiez.}
\color{reponse}
réponse réponse réponse réponse réponse réponse réponse réponse réponse réponse réponse réponse réponse réponse réponse réponse réponse réponse réponse réponse réponse réponse réponse réponse réponse réponse réponse réponse réponse réponse réponse réponse réponse réponse réponse réponse réponse réponse réponse réponse réponse réponse réponse réponse réponse réponse réponse réponse réponse réponse réponse réponse réponse 
\color{black}



\renewcommand{\questionA}{Dans TCP, comment fixe-t-on les numéros des premiers segments transmis dans chaque
sens d’une connexion ? } 
\renewcommand{\questionB}{Si l’on attribuait systématiquement la valeur 0 (par exemple) à ces premiers numéros, quel serait le risque et comment pourrait-on l’éviter en conservant toutefois cette numérotation ? Quel serait l’inconvénient ? }
\subsubsection[\questionA \questionB]{
\begin{enumerate}[(a)]
	\item \questionA
    \item \questionB
\end{enumerate}}
\color{reponse}
réponse réponse réponse réponse réponse réponse réponse réponse réponse réponse réponse réponse réponse réponse réponse réponse réponse réponse réponse réponse réponse réponse réponse réponse réponse réponse réponse réponse réponse réponse réponse réponse réponse réponse réponse réponse réponse réponse réponse réponse réponse réponse réponse réponse réponse réponse réponse réponse réponse réponse réponse réponse réponse 
\color{black}



\renewcommand{\questionA}{Dans quelle(s) situation(s) le protocole de routage à vecteur de distances (DV) risque-t-il de ne pas converger ? } 
\renewcommand{\questionB}{Décrivez un comportement pathologique possible à l’aide d’un exemple simple. } 
\renewcommand{\questionC}{Comment peut-on atténuer ce phénomène ?}
\subsubsection[\questionA \questionB \questionC]{
\begin{enumerate}[(a)]
	\item \questionA
    \item \questionB
    \item \questionC
\end{enumerate}}
\color{reponse}
réponse réponse réponse réponse réponse réponse réponse réponse réponse réponse réponse réponse réponse réponse réponse réponse réponse réponse réponse réponse réponse réponse réponse réponse réponse réponse réponse réponse réponse réponse réponse réponse réponse réponse réponse réponse réponse réponse réponse réponse réponse réponse réponse réponse réponse réponse réponse réponse réponse réponse réponse réponse réponse 
\color{black}



\renewcommand{\questionA}{Décrivez les principes du protocole de routage inter-domaine BGP. } 
\renewcommand{\questionB}{Expliquez comment BGP permet à un réseau périphérique (« stub ») multi-connecté (« multihomed
») de ne pas accepter du trafic de transit. }
\subsubsection[\questionA \questionB]{
\begin{enumerate}[(a)]
	\item \questionA
    \item \questionB
\end{enumerate}}
\color{reponse}
réponse réponse réponse réponse réponse réponse réponse réponse réponse réponse réponse réponse réponse réponse réponse réponse réponse réponse réponse réponse réponse réponse réponse réponse réponse réponse réponse réponse réponse réponse réponse réponse réponse réponse réponse réponse réponse réponse réponse réponse réponse réponse réponse réponse réponse réponse réponse réponse réponse réponse réponse réponse réponse 
\color{black}



\renewcommand{\questionA}{Déterminez analytiquement l’expression de l’efficacité du protocole ALOHA discrétisé (slotted ALOHA) en fonction de la charge du réseau pour un grand nombre de stations actives. On supposera que chaque station émet dans un slot avec une probabilité p. } 
\renewcommand{\questionB}{Représentez l’efficacité graphiquement (avec définition des axes), et expliquez la forme de la
courbe. } 
\renewcommand{\questionC}{La suppression des slots (Cf. ALOHA pur) améliore-t-elle les performances ? Pourquoi ?}
\subsubsection[\questionA \questionB \questionC]{
\begin{enumerate}[(a)]
	\item \questionA
    \item \questionB
    \item \questionC
\end{enumerate}}
\color{reponse}
réponse réponse réponse réponse réponse réponse réponse réponse réponse réponse réponse réponse réponse réponse réponse réponse réponse réponse réponse réponse réponse réponse réponse réponse réponse réponse réponse réponse réponse réponse réponse réponse réponse réponse réponse réponse réponse réponse réponse réponse réponse réponse réponse réponse réponse réponse réponse réponse réponse réponse réponse réponse réponse 
\color{black}
